\section{Projektorganisation}

\subsection{Stakeholder}

\paragraph*{}
Die Stakeholder sind �bersichtlich im Stakeholder Register in Tabelle 1 dargestellt.

\begin{table}
\caption{Stakeholder Register}
\begin{tabular}[h]{|p{3,5cm}|p{4,7cm}|p{5,8cm}|}
\hline
\textsc{Name} & \textsc{Kontakdaten} & \textsc{Rolle}\\
\hline
FHDW Studenten und Lehrende & all@fhdw.de & Nutzer Bibliothek\\
\hline
F�rster, Ulrich & Ulrich.Foerster@fhdw.de & Leiter FHDW\\
\hline
Gaber, Martin & Martin.Gaber@bib.de & Hard- und Software\\
\hline
Hommes, Thomas & Hommes@bg.bib.de & Ansprechpartner Lotus Notes\\
\hline
Kutschke, Heike & Heike.Kutschke@fhdw.de & Ausgabe Zugangskarten, Medien mit ID versehen\\
\hline
Lachnit, Winfried & Winfried.Lachnit@fhdw.de & Auftraggeber\\
\hline
Platzhalter & noch zu kl�ren & Lieferant Hardware\\
\hline
\end{tabular}
\end{table}

\subsection{Organisationsform}

\paragraph*{}
Da alle Projektbeteiligten, zus�tzlich zu ihren Hauptaufgaben, nur zeitlich begrenzt am Projekt mitarbeiten und die eigentliche Aufbauorganisation erhalten bleibt, wird die Einflu�projektorganisation gew�hlt.

\subsection{Projektrollen}

\paragraph*{}
Projekteigner Lieferant
Projektleiter Lieferant
Projektsponsor Kunde
Auftraggeber Kunde

Auftraggeber ist Herr Winfried Lachnit, er �bernimmt auch das Testen in allen Phasen des Projekts.
Frau Kutschke �bernimmt die Ausgabe der Zugangskarten und die Versehen der Medien mit den IDs.
Projektsponsor

\subsection{Verantwortlichkeiten}

\paragraph*{}
Die Verantwortlichkeiten der Team-Mitglieder sind �bersichtlich in Tabelle 2 dargestellt.

\begin{table}
\caption{Verantwortlichkeiten}
\begin{tabular}[h]{|p{6cm}|p{9cm}|}
\hline
\textsc{Name} & \textsc{Verantwortlichkeiten}\\
\hline
Gaber, Martin & Hard- und Software\\
\hline
Hommes, Thomas & Ansprechpartner Lotus Notes\\
\hline
Kutschke, Heike & Ausgabe Zugangskarten, Medien mit ID versehen\\
\hline
Lachnit, Winfried & Auftraggeber, Abnahme Endprodukt\\
\hline
\end{tabular}
\end{table}