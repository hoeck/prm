\section{Allgemeine Projektbeschreibung}

\subsection{Projektthema}

\paragraph*{}
In dem innerbetrieblichen Projekt "Personalfreier Zugang und Ausleihe in der Bibliothek der FHDW" soll der personalfreie Zugang und die personalfreie Ausleihe von Medien (B�cher/CDs/DVDs/Zeitungen...) in der gerade entstehenden Bibliothek der FHDW Dresden realisiert werden. Vorschl�ge seitens der Auftraggeber sind die L�sung mittels der RFID-Technologie per Gate oder Terminal. 
Als Anschauungsobjekt dient zum Beispiel die Bibliothek der HTW Dresden.

\subsection{Anwendungsgebiet}

\paragraph*{}
Ziele dieses Projekts sind der personalfreie Zugang zu einer kleinen Bibliothek und die personalfreie Ausleihe der Medien sowie gleichzeitige �bertragung der Daten an ein schon existierende Lotus Notes System.
Hierbei soll die ID des Nutzers �ber eine Karte ermittelt werden. Nur f�r autorisierte Personen soll der Zugang zur Bibliothek m�glich sein. Den Medien sollen ebenfalls eine ID zugewiesen werden. Die ID des Nutzers, soll mit der ID des Mediumsverkn�pft, ans System �bertragen werden (Anbindung am Lotus Notes).

\subsection{Projektgr��e}

\paragraph*{}
Das Budget ist mit 10000 Euro veranschlagt. Die vorhandenen Ressourcen werden soweit wie m�glich genutzt. Personal Computer und Server, sowie ein Internetzugang �ber Wireless Local Area Network sind als gegeben zu betrachten. Ben�tigte Mitarbeiter werden f�r die Zeit des Projektes, soweit m�glich, freigestellt.

\subsection{Ziele}

\paragraph*{}
Ziel ist es ein stabiles System aufzustellen, das sowohl mechanischen als auch informationstechnischen Angriffen standh�lt. Der Aufwand f�r die Beteiligten am laufenden System soll gering gehalten werden. Die Betonung liegt auf der Einfachheit und Kosteng�nstikeit des Systems.

\paragraph*{}
Es wird nicht zus�tzlich f�r das Lotus Notes System programmiert. Die Registrierung, Verwaltung und Suche wird von Lotus Notes �bernommen. Nur die Anbindung an das schon existierende System wird genutzt oder bei Bedarf erstellt. Zus�tzliche Funktionalit�ten der Software sind nicht Bestandteil des Projekts. Die derzeitigen Sicherheitsvorkehrungen bleiben unver�ndert. Die R�ume werden nachts abgeschlossen und die Alarmanlage wird eingeschaltet. Gro�e restriktive Vorkehrungen, wie zum Beispiel eine Sicherheitsschleuse, wird es nicht geben. Es wird somit keine M�glichkeit geben die Bibliothek rund um die Uhr zu besuchen. 